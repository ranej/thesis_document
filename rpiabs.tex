%%%%%%%%%%%%%%%%%%%%%%%%%%%%%%%%%%%%%%%%%%%%%%%%%%%%%%%%%%%%%%%%%%% 
%                                                                 %
%                            ABSTRACT                             %
%                                                                 %
%%%%%%%%%%%%%%%%%%%%%%%%%%%%%%%%%%%%%%%%%%%%%%%%%%%%%%%%%%%%%%%%%%% 
 
\specialhead{ABSTRACT}
 
Large eddy simulation (LES) is an attractive turbulence modeling approach due to the balance it provides between computational cost and turbulence/scale-resolving capabilities. 
LES is shown to be robust and provides accurate predictions for complex flow problems including unsteady aerodynamic flows with spatiotemporal inhomogeneity.
In this work, the specific problem of interest is of flow over surging airfoils, which arise in many aerodynamic problems of interest. For example, for a rotorcraft in a forward flight, or for wind turbines in non-uniform flows (e.g., shear).

Mesh resolution requirements for such complex and unsteady flow problems are not known \textit{a priori}.
In this work, we develop an adaptive approach where mesh resolution is refined or adapted based on an \textit{a posteriori} error estimator leading to error-driven/controlled adaptive LES.
In addition, a flow feature-based adaptation criterion that can use the error estimator is developed.
Both the LES methodology and the error estimator used here are based on the variational multiscale (VMS) framework.

A range of Reynolds numbers and advance ratios is considered in adaptive LES for surging airfoils.
Three different adaptive strategies based on VMS error estimator are explored: (i) zonal-based refinement/adaptation, (ii) nodal size field-based adaptation, and (iii) feature-based refinement/adaptation.
The zonal-based strategy is found to be most effective. This strategy is applied for adaptive LES of flow over surging airfoils to construct a series of adapted meshes and mesh convergence is shown. In particular, for quantities of interest including pressure coefficient and leading edge vortex (LEV) evolution.
In addition, very high advance ratios are considered, where massive trailing edge separation also occurs due to flow reversal. In these cases, LES includes active flow control in the form of active reflex camber that is applied to dynamically morph the airfoil shape during flow reversal (i.e., only for a part of the surging cycle). Active reflex camber results in a significant reduction in drag force and its fluctuations.






