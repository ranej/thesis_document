%%%%%%%%%%%%%%%%%%%%%%%%%%%%%%%%%%%%%%%%%%%%%%%%%%%%%%%%%%%%%%%%%%% 
%                                                                 %
%                            ABSTRACT                             %
%                                                                 %
%%%%%%%%%%%%%%%%%%%%%%%%%%%%%%%%%%%%%%%%%%%%%%%%%%%%%%%%%%%%%%%%%%% 
 
\specialhead{ABSTRACT}
 
Large eddy simulation (LES) is an attractive turbulence modeling approach due to the balance it provides between computational cost and turbulence/scale-resolving capabilities. 
LES is shown to be robust and provides accurate predictions for complex flow problems including aerodynamic problems with spatiotemporal inhomogeneity.
In this work, the specific problem of interest is of flow over surging airfoils.
Surging motion arises in many aerodynamic problems. For example, for a rotorcraft in forward flight, or for wind turbines in non-uniform flows (e.g., shear).

%TODO: Check italics for a priori/posteriori

Mesh resolution requirements for such complex and unsteady flow problems are not known \textit{a priori}.
In this work, we develop an adaptive approach where mesh resolution is refined or adapted based on an \textit{a posteriori} error estimator leading to error-drive/controlled adaptive large eddy simulations.
Both the LES methodology and the error estimator used here are based on the variational multiscale (VMS) framework.

LES is performed for flow over a surging airfoil for various Reynolds numbers and advance ratios.
Adaptive LES approach is applied to surging airfoils.
Three different adaptive strategies are explored: (i) zonal-based refinement/adaptation, (ii) nodal size field-based adaptation, and (iii) feature-based refinement/adaptation.
It is shown that zonal-based refinement/adaptation is the best strategy for error-estimator driven mesh adaptation, and this strategy is applied to LES of flow over a surging airfoil to construct a series of meshes. For these meshes, convergence is shown, in particular for quantities such as pressure coefficient, LEV velocity profiles, etc.

%TODO: SOmething else instead of best strategy

%TODO: very high advance ratio, trailing edge separation, 

For surging airfoils at very high advance ratios, where massive trailing edge separation occurs due to flow reversal, active flow control in the form of active reflex camber is applied to dynamically morph the airfoil shape during flow reversal. This results in significant reduction of drag force, along with reduction in drag force fluctuations.



