%%Cp plots

In this section, we focus on $C_p$ for the different meshes obtained from the zonal-based refinement.
Here, phase- and spanwise-averaged data is considered over multiple cycles.

A comparison of $C_p$ for the different meshes is shown in Figure \ref{fig:zonal_Cp_plots_LEV}. This is done for 4 phases of interest leading up to the LEV formation. 
For $\psi=150^\circ$, flow has started to separate for Mza2 and Mza3 meshes around $x/c = 0.2$, whereas M0 and Mza1 meshes fail to capture this flow separation. 
Mza2 and Mza3 meshes agree well with each other. 
Note that meshes with the same in-plane resolution does not show significant variations when using different spanwise resolutions.

 
For $\psi=180^\circ$, M0 mesh does not capture any peak in $C_p$.
Mza1, Mza2 and Mza3 meshes show shear layer roll-up with a peak in $C_p$ around $x/c = 0.1$.
Note that the peak in $C_p$ corresponds with the low pressure associated with a vortex roll-up.
Mza1 meshes show a lower peak in $C_p$ than Mza2 and Mza3 meshes. 
Flow separation is also not clearly visible in spanwise vorticity plots for Mza1 meshes shown in Figure \ref{fig:vorticity_zonal_180}, as compared to Mza2 and Mza3 meshes.
Mza2 and Mza3 meshes agree well with each other.


For $\psi=210^\circ$, a peak in $C_p$ is observed for all meshes apart from M0 mesh.
The peak and drop in $C_p$ occurs between $x/c=0.15$ and $x/c=0.22$ for Mza1 meshes. 
For Mza2 and Mza3 meshes, this occurs between $x/c=0.25$ and $x/c=0.35$. Mza1 meshes predict an overall higher $C_p$ value than Mza2 and Mza3 meshes.
Mza2 and Mza3 meshes show some differences, with Mza2 mesh predicting a slightly higher $C_p$, along with the location of the peak at a higher $x/c$ location than Mza3 meshes.

For $\psi=240^\circ$, a peak in $C_p$ can be seen for all meshes.
M0 mesh predicts a peak and drop in $C_p$ between $x/c=0.1$ and $x/c=0.2$, whereas Mza1 meshes predict a peak and drop in $C_p$ around $x/c=0.13$ and $x/c=0.25$. 
Mza2 and Mza3 meshes predict a peak and drop in $C_p$ around $x/c=0.17$ and $x/c=0.3$. 
Once again, it is observed that Mza2 and Mza3 meshes show a reasonable agreement, and meshes with the same in-plane resolution does not show significant variations when using different spanwise resolutions.

\begin{figure}[H]
\centering

\begin{subfigure}[b]{0.475\textwidth}
	\centering
	\includegraphics[width=1\textwidth]{figures/zonal_adapt_results/Cp/phase_150.png}
	\caption{ $C_p$ at $\psi$ = $150^\circ$}
	\label{fig:zonal_Cp_150}
\end{subfigure}
\begin{subfigure}[b]{0.475\textwidth}
\centering
\includegraphics[width=1\textwidth]{figures/zonal_adapt_results/Cp/phase_180.png}
\caption{ $C_p$ at $\psi$ = $180^\circ$}
\label{fig:zonal_Cp_180}
\end{subfigure}
\begin{subfigure}[b]{0.475\textwidth}
\centering
\includegraphics[width=1\textwidth]{figures/zonal_adapt_results/Cp/phase_210.png}
\caption{ $C_p$ at $\psi$ = $210^\circ$}
\label{fig:zonal_Cp_210}
\end{subfigure}
\begin{subfigure}[b]{0.475\textwidth}
\centering
\includegraphics[width=1\textwidth]{figures/zonal_adapt_results/Cp/phase_240.png}
\caption{ $C_p$ at $\psi$ = $240^\circ$}
\label{fig:zonal_Cp_240}
\end{subfigure}
\caption{$C_p$ comparison for different meshes. Top surface $C_p$ is denoted by solid lines and bottom surface $C_p$ is denoted by dashed lines}
\label{fig:zonal_Cp_plots_LEV}
\end{figure}

Since we have already seen through spanwise vorticity and $Cp$ that the spanwise resolution in each mesh series does not affect the solution significantly, for brevity we only show the data from meshes with the second highest spanwise resolution. 
$C_p$ plots at $\psi=270^\circ$ and $\psi=300^\circ$, which is shown in Figure \ref{fig:zonal_Cp_plots_TEV}, show formation of trailing edge separation, which is evident from peaks in $C_p$ near the geometric trailing edge. 
For both these phases, $C_p$ for all meshes compare well with each other apart from the M0 mesh.


\begin{figure}[H]
\begin{subfigure}[b]{0.475\textwidth}
\centering
\includegraphics[width=1\textwidth]{figures/zonal_adapt_results/Cp/phase_270.png}
\caption{ $C_p$ at $\psi$ = $270^\circ$}
\label{fig:zonal_Cp_270}
\end{subfigure}
\begin{subfigure}[b]{0.475\textwidth}
\centering
\includegraphics[width=1\textwidth]{figures/zonal_adapt_results/Cp/phase_300.png}
\caption{ $C_p$ at $\psi$ = $300^\circ$}
\label{fig:zonal_Cp_300}
\end{subfigure}
\caption{$C_p$ comparison for different meshes. Top surface $C_p$ is denoted by solid lines and bottom surface $C_p$ is denoted by dashed lines}
\label{fig:zonal_Cp_plots_TEV}
\end{figure}

