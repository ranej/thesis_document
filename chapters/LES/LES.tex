This work uses the incompressible Navier Stokes equations in the arbitrary Lagrangian Eulerian (ALE) description.
The strong form of the equations is given as

\begin{equation}
\label{eq:NS_strong}
\begin{split}
  &u_{k,k} = 0\\
  &u_{i,t} + (u_j - u^m_j) u_{i,j}  =  - p_{,i} + \tau^{\nu}_{ij,j}  + f_i 
\end{split}
\end{equation}

\noindent where $u_i$ is the velocity vector,
$u^m_i$ is the mesh velocity vector,
$p$ is the pressure (scaled by the constant density),
$\tau^{\nu}_{ij}=2 \nu S_{ij}$ is the symmetric (Newtonian) viscous stress
tensor (scaled by the density), $\nu$ is the kinematic viscosity, 
$S_{ij}=0.5(u_{i,j} + u_{j,i})$ is the strain-rate tensor,
and $f_i$ is the body force vector (per unit mass).
Note that Einstein summation notation is used.

The weak form is stated as follows:
find $\mathbf{u} \in \bm{\mathcal{S}}$ and $p \in \mathcal{P}$ such that

\begin{equation}
\label{eq:NS_weak}
\begin{split}
B(\{w_i,q\}, \{u_i,p\};u^m_l) =& \int\limits_\Omega [w_i (u_{i,t} + u_i u^m_{j,j}) + w_{i,j} (-u_i (u_j - u^m_j) + \tau^{\nu}_{ij} - p \delta_{ij}) -q_{,k} u_k] \, d\Omega \\
      +& \int\limits_{\Gamma_h} [w_i (u_i(u_j-u^m_j)  - \tau^{\nu}_{ij} + p \delta_{ij}) n_j + q u_k n_k] \, d\Gamma_h \\
	  =& \int\limits_\Omega w_i f_i d\Omega
\end{split}
\end{equation}

\noindent for all $\mathbf{w} \in \bm{\mathcal{W}}$ and $q \in \mathcal{P}$.
$\bm{\mathcal{S}}$ and $\mathcal{P}$ are suitable trial/solution spaces
and $\bm{\mathcal{W}}$ is the test/weight space.
$\mathbf{w}$ and $q$ are the weight
functions for the velocity and pressure variables, respectively.
$\Omega$ is the spatial domain and $\Gamma_h$ is the portion of the domain
boundary with Neumann or natural boundary conditions.

The above weak form can be written concisely as:
find $\bm{U} \in \bm{\mathcal{U}}$ such that 

\begin{equation}
\label{eq:NS_weak2}
B(\bm{W},\bm{U};u^m_l) = (\bm{W},\bm{F})
\end{equation}

\noindent for all $\bm{W}=[\mathbf{w},q]^T \in \bm{\mathcal{V}}$.
$\bm{U}=[\mathbf{u},p]^T$ is the vector of unknown solution variables
and $\bm{F}=[\mathbf{f},0]^T$ is the source vector.
The solution and weight spaces are:
$\bm{\mathcal{U}} = \{\bm{U}=[\mathbf{u},q]^T | \mathbf{u} \in 
\bm{\mathcal{S}}; p \in \mathcal{P}\}$ and
$\bm{\mathcal{V}} = \{\bm{W}=[\mathbf{w},q]^T | \mathbf{w} \in
\bm{\mathcal{W}}; q \in \mathcal{P}\}$, respectively.

Throughout this text $B(\cdot,\cdot)$ is used to represent the semi-linear
form that is linear in its first argument and $(\cdot,\cdot)$ is used to denote the $L_2$
inner product.
$B(\bm{W},\bm{U};u^m_l)$ is split into bilinear and semi-linear terms as shown below. 

\begin{equation}
B(\bm{W},\bm{U};u^m_l) = B_1(\bm{W},\bm{U};u^m_l) + B_2(\bm{W}, \bm{U}) = (\bm{W},\bm{F})
\end{equation}

\noindent where $B_1(\bm{W},\bm{U};u^m_l)$ contains the bilinear terms and $B_2(\bm{W}, \bm{U})$ consists of the semi-linear terms. These are defined as

\begin{equation}
\begin{split}
B_1(\bm{W},\bm{U};u^m_l)  =& \int\limits_\Omega [w_i (u_{i,t} + u_i u^m_{j,j}) + w_{i,j} (u_i u^m_j + \tau^{\nu}_{ij} - p \delta_{ij}) -q_{,k} u_k] \, d\Omega \\
      +& \int\limits_{\Gamma_h} [w_i (-u_i u^m_j -\tau^{\nu}_{ij} + p \delta_{ij}) n_j + q u_k n_k] \, d\Gamma_h 
\end{split}
\end{equation}

\begin{equation}
B_2(\bm{W},\bm{U})  = -\int\limits_\Omega w_{i,j} u_i u_j d\Omega +\int\limits_{\Gamma_h} w_i u_i u_j n_j d\Gamma_h 
\end{equation}



The Galerkin weak form is obtained by considering
the finite-dimensional or discrete solution spaces
$\bm{\mathcal{S}}^h \subset \bm{\mathcal{S}}$ and
$\mathcal{P}^h \subset \mathcal{P}$ and the weight
space $\bm{\mathcal{W}}^h \subset \bm{\mathcal{W}}$, where
the superscript $h$ is used as a mesh parameter to denote discretized spaces and variables in a finite element context.
Using these spaces, $\bm{\mathcal{U}}^h =
\{ \bm{U}^h=[\mathbf{u}^h,p^h]^T | \mathbf{u}^h \in \bm{\mathcal{S}}^h; p^h \in \mathcal{P}^h \}$ and $\bm{\mathcal{V}}^h =
\{ \bm{W}^h=[\mathbf{w}^h,q^h]^T | \mathbf{w}^h \in \bm{\mathcal{W}}^h; q^h \in \mathcal{P}^h \}$ are defined.
The Galerkin weak form is then stated concisely as:
find $\bm{U}^h \in \bm{\mathcal{U}}^h$ such that

\begin{equation}
\label{eq:NS_Gal_weak}
B(\bm{W}^h,\bm{U}^h) = (\bm{W}^h,\bm{F})
\end{equation}

\noindent for all $\bm{W}^h \in \bm{\mathcal{V}}^h$.
Note for brevity we have dropped $u^m_l$ term in the arguments of the semi-linear form.
The Galerkin weak formulation corresponds to a method for
direct numerical simulation since no modeling is employed.
However, when the finite-dimensional spaces are incapable of representing the
fine/small scales, the Galerkin formulation yields an
inaccurate solution.
A model term is added to overcome this difficulty, e.g.,
as done in the residual-based variational multiscale (RBVMS) formulation.

In RBVMS, a set of model terms is added to the Galerkin weak form that results in
the following variational formulation:
find $\bm{U}^h \in \bm{\mathcal{U}}^h$ such that

\begin{equation}
\label{eq:NS_rbvms}
 B(\bm{W}^h,\bm{U}^h) + M_{rbvms}(\bm{W}^h,\bm{U}^h ) = (\bm{W}^h,\bm{F})
\end{equation}

\noindent for all $\bm{W}^h \in \bm{\mathcal{V}}^h$.
$M_{rbvms}$ represents the set of model terms
due to the RBVMS approach.

A scale separation is used to decompose
the solution and weight spaces as
$\bm{\mathcal{S}}=\bm{\mathcal{S}}^h \oplus \bm{\mathcal{S}}'$ and $\mathcal{P} = \mathcal{P}^h \oplus \mathcal{P}'$,
and $\bm{\mathcal{W}}=\bm{\mathcal{W}}^h\oplus \bm{\mathcal{W}}'$,
respectively. Thus, the solution and weight functions
are decomposed as $u_i = u^h_i + u'_i$ and $p = p^h + p'$ or $\bm{U}=\bm{U}^h+\bm{U}'$, and
$w_i = w^h_i + w'_i$ and $q = q^h + q'$ or $\bm{W}=\bm{W}^h+\bm{W}'$, respectively.
Note that coarse-scale or resolved quantities are
denoted by $(\cdot)^h$ and fine-scale or unresolved quantities 
by $(\cdot)'$.
The coarse-scale quantities are resolved by
the grid whereas the effects of the fine scales on the coarse scales
are modeled.
In RBVMS, the fine scales are modeled as a function of the strong-form residual
due to the coarse-scale solution. This is represented abstractly as
$\bm{U}' = \mathcal{F}(\bm{R}(\bm{U}^h);\bm{U}^h)$, where
$\bm{R}(\cdot) = [\bm{R}^m(\cdot),R^c(\cdot)]^T$ 
is the strong-form residual of the equations with
$\bm{R}^m(\cdot)$ (or $R^m_i(\cdot)$) and $R^c(\cdot)$ as those
of the momentum and continuity equations, respectively.
Specifically, the fine-scale quantities are modeled as 
$u'_i \approx -\tau_M R^m_i(u_k^h,p^h;u^m_l)$
and $p' \approx -\tau_C R^c(u^h_k)$, where
$\tau_C$ and $\tau_M$ are stabilization parameters
(e.g., see details in Tran and Sahni~\cite{bib:tran2017b}).
This provides a closure to the coarse-scale problem as it involves coarse-scale solution as the only unknown. 
This is why $M_{rbvms}(\bm{W}^h,\bm{U}^h )$ is written only in terms of the unknown coarse-scale solution $\bm{U}^h$.
In summary, $M_{rbvms}(\bm{W}^h,\bm{U}^h)$ can be written as

%\begin{equation}
%\label{eq:NS_Mrbvms}
%\begin{split}
%M_{rbvms}(\bm{W}^h,\bm{U}^h)= \sum_e \int\limits_{\Omega^h_e} [&\underbrace{-(w^h_i u^m_{j,j}
%+ w^h_{i,j} u^m_j ) \tau_M {R}^m_i(u_k^h,p^h;u^m_l)}_{M^{ALE}_{rbvms}(\bm{W}^h,\bm{U}^h)} \\
%&+ \underbrace{q^h_{,i} \tau_M {R}^m_i(u_k^h,p^h;u^m_l)}_{M_{rbvms}^{cont}(\bm{W}^h, \bm{U}^h)} + \underbrace{w^h_{i,j}  \tau_C R^c(u^h_k)\delta_{ij}}_{M^P_{rbvms}(\bm{W}^h, \bm{U}^h)} \\
%&+\underbrace{w^h_{i,j}\left( u^h_i \tau_M {R}^m_j(u_k^h,p^h;u^m_l) + \tau_M {R}^m_i(u_k^h,p^h;u^m_l) u^h_j \right)}_{M^C_{rbvms}(\bm{W}^h, \bm{U}^h)} \\
%&\underbrace{-w^h_{i,j} \tau_M {R}^m_i(u_k^h,p^h;u^m_l) \tau_M {R}^m_j(u_k^h,p^h)}_{M^R_{rbvms}(\bm{W}^h,\bm{U}^h)} ]d\Omega^h_e
%\end{split}
%\end{equation}

%%%\begin{equation*}
%%%M_{rbvms}(\bm{W}^h,\bm{U}^h)=
%%%\end{equation*}

\begin{equation}
\label{eq:NS_Mrbvms}
\begin{split}
M_{rbvms}(& \bm{W}^h, \bm{U}^h) = \\
\sum_e \int\limits_{\Omega^h_e} [&\underbrace{-(w^h_i u^m_{j,j}
+ w^h_{i,j} u^m_j ) \tau_M {R}^m_i(u_k^h,p^h;u^m_l)}_{M^{ALE}_{rbvms}(\bm{W}^h,\bm{U}^h)} \\
&+ \underbrace{q^h_{,i} \tau_M {R}^m_i(u_k^h,p^h;u^m_l)}_{M_{rbvms}^{cont}(\bm{W}^h, \bm{U}^h)} + \underbrace{w^h_{i,j}  \tau_C R^c(u^h_k)\delta_{ij}}_{M^P_{rbvms}(\bm{W}^h, \bm{U}^h)} \\
&+\underbrace{w^h_{i,j}\left( u^h_i \tau_M {R}^m_j(u_k^h,p^h;u^m_l) + \tau_M {R}^m_i(u_k^h,p^h;u^m_l) u^h_j \right)}_{M^C_{rbvms}(\bm{W}^h, \bm{U}^h)} \\
&\underbrace{-w^h_{i,j} \tau_M {R}^m_i(u_k^h,p^h;u^m_l) \tau_M {R}^m_j(u_k^h,p^h)}_{M^R_{rbvms}(\bm{W}^h,\bm{U}^h)} ]d\Omega^h_e
\end{split}
\end{equation}



Note that all model terms are written in terms of the resolved scales within
each element (where $e$ denotes an element and contributions from all elements
are summed).
The last model term is used to represent the Reynolds stresses
(i.e., $M^R_{rbvms}$)
while the two terms prior to it are used to represent the cross-stress terms
(i.e., $M^C_{rbvms}$).

In previous studies~\cite{bib:wang2010,bib:tran2017a}, it was found that the RBVMS model provides a good
approximation for the turbulent dissipation due to the cross stresses but the
dissipation due to the Reynolds stresses is under predicted and turns out to be
insufficient. Therefore, a combined subgrid-scale model was employed which
uses the RBVMS model for the cross-stress terms
and the dynamic Smagorinsky eddy-viscosity model for the Reynolds stress terms.
This was done in both 
a finite element code~\cite{bib:tran2016,bib:tran2017a}
and a spectral code~\cite{bib:wang2010}.
The combined subgrid-scale model is defined as

\begin{equation}
\label{eq:NS_comb}
 B(\bm{W}^h,\bm{U}^h) + M_{comb}(\bm{W}^h,\bm{U}^h; C_S, h)  = (\bm{W}^h,\bm{F})
\end{equation}

\noindent where

\begin{equation}
\label{eq:NS_Mcomb}
\begin{split}
M_{comb}(\bm{W}^h,\bm{U}^h; C_S, h) =& M^{ALE}_{rbvms}(\bm{W}^h,\bm{U}^h) +M^{cont}_{rbvms}(\bm{W}^h, \bm{U}^h) \\
& + M^P_{rbvms}(\bm{W}^h, \bm{U}^h) + M^C_{rbvms}(\bm{W}^h, \bm{U}^h) \\
& + (1-\gamma) M^R_{rbvms}(\bm{W}^h, \bm{U}^h)
\\
& +\gamma M^R_{smag}(\bm{W}^h,\bm{U}^h; C_S, h)
\end{split}
\end{equation}

\begin{equation}
\label{eq:NS_Smag_sub}
%M^R_{smag}(\bm{W}^h,\bm{U}^h; C_S, h) = \int\limits_\Omega w^h_{i,j} 2 \nu_t S^h_{ij} d\Omega = \int\limits_\Omega w^h_{i,j} 2 \, \underbrace{(C_S h)^2 |S^h|}_{\nu_t} S^h_{ij} d\Omega
M^R_{smag}(\bm{W}^h,\bm{U}^h; C_S, h) = \int\limits_\Omega w^h_{i,j} 2 \, \underbrace{(C_S h)^2 |S^h|}_{\nu_t} S^h_{ij} d\Omega
\end{equation}

\noindent where $\nu_t$ is the eddy viscosity, $|S^h|$ is the norm of the
strain-rate tensor (i.e., $|S^h|=\sqrt{2 \bm{S}^h : \bm{S}^h} = \sqrt{2 S^h_{ij} S^h_{ij}}$),
$h$ is the local mesh size,
and $C_S$ is the Smagorinsky parameter.
The parameter $\gamma$ is set to be either $0$ or $1$ in order to control
which model is used for the Reynolds stresses.
Note that $\gamma=0$ results in the
original RBVMS model and $\gamma=1$ results in the combined subgrid-scale model.
In this study, $\gamma=1$ is employed.
The Smagorinsky parameter is computed dynamically in a local fashion as discussed below.
