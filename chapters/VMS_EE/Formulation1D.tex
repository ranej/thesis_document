

The steady advection-diffusion equation with the strong form given by: find the scalar $\psi$ such that

\begin{equation}
\mathcal{L}(\psi) = a_i \psi_{,i} - \kappa \psi_{,ii} = f
\end{equation}

\noindent where $a_i$ is an advection vector, $\kappa$ is the diffusivity constant and $f$
is a forcing function.
We can define the corresponding spaces for the solution and weight functions as usual

\begin{equation}
\label{eq:testspace}
\mathcal{S}_{\psi} = \{ \psi| \psi \in H^1(\Omega)^N, \psi = g \text{ on } \Gamma_g\} 
\end{equation}

\begin{equation}
\label{eq:Vweightspace}
\mathcal{W}_{\psi} = \{ w|w \in H^1(\Omega)^N, w = 0 \text{ on } \Gamma_g\}
\end{equation}
\noindent where $\Gamma_g$ and $\Gamma_h$ are portions of domain boundary with Dirichlet and Neumann boundary conditions, respectively.

\noindent The corresponding weak form is given by: find $\psi \in \mathcal{S}_{\psi}$ such that

\begin{equation} 
\int_\Omega \left[ w a_i \psi_{,i} + \kappa {w}_{,i}\psi_{,i}\right] \, d\Omega- 
\int_{\Gamma_h} \kappa w \psi_{,i} n_i\, d\Gamma_h = 
\int_\Omega w f \, d\Omega, \qquad \forall w \in \mathcal{W}_{\psi}
\label{eq:AD_model}
\end{equation}

Equation \ref{eq:uPrimeSoln} describes the relationship between the coarse-scale solution $\bar{\psi}$ and the discretization error.
The strong residual acts as a local source for error and the fine-scale Green's function distributes it over the spatial domain.
%Following the optimality approximation of $\bar{u}$ in the $\HOne$-sense, we consider only the $\HOne$-seminorm.
From Equation \ref{eq:uPrimeSoln}, we can define the $\HOne$-seminorm over an element $k$

\begin{equation}
    |e^{\psi}_k|_* = |e^{\psi}_k|_{\HOne(\Omega^h_k)} = |\psi'_k|_{\HOne(\Omega^h_{k})}= |(\mathcal{R}(\bar{\psi}_k)| \norm{ \int_{\Omega^{h}_{k}} g_{,i}' d\Omega^h_{k}}_{L^2(\Omega^h_k)}
\end{equation}

\noindent where $\mathcal{R}(\bar{\psi})=\mathcal{L}(\bar{\psi})-f=\mathcal{L}(\psi^h)-f$ is constant for piecewise linear finite elements with piecewise constant input data and material parameters/properties.
We focus on the $\HOne$-seminorm in particular since we want to control the error in the gradients in the solution field.

For the 1D advection-diffusion equation, the fine-scale Green's function has an analytical form (see Hauke, Doweidar and Miana \cite{hauke2006proper}) that is used to attain the following simplified form

\begin{align}
      \norm{\psi_k'}_{H^1(\Omega^h_k)} &= \nu^{err}_{k,1D} |\mathcal{R}(\bar{\psi}_k)| \sqrt{|\Omega^h_k|} \\
            &= \nu^{err}_{k,1D} \norm{\mathcal{R}(\bar{\psi}_k)}_{L^2(\Omega^h_k)}
      \label{eq:VMSerr1D}
\end{align}

%\noindent where $\nu^{err}_{e,1D} = \dfrac{1}{|a|} min \left( \sqrt{\Peh},\dfrac{\Peh}{\sqrt{3}} \right)$ and $\Peh = \frac{h_e |a|}{2\kappa}$ is the cell-Peclet number.
\noindent where $\nu^{err}_{k,1D} = \dfrac{1}{|a_i|} \sqrt{\Peh \text{coth}({\Peh})-1}$ and $\Peh = \frac{h_k |a_i|}{2\kappa}$ is the cell-Peclet number, and $h_k$ denotes the local mesh size.

Note that this is an explicit expression which does not require solving additional differential equations (local or global) like implicit error estimators.
This makes Equation \ref{eq:VMSerr1D} inexpensive to evaluate, although limited in scope to 1D cases. The effectivity of the VMS based error estimator was tested for a series of cell Peclet numbers $Pe_h$ by Zhang \cite{zhang19} and showed that across all cases, the VMS error estimator provides the exact error in 1D. 
This is expected since the VMS approach in 1D for a linear, steady advection-diffusion case with constant material parameters makes no assumptions or approximations.

\subsection{Generalization to Multi-dimensional Cases}

The primary component of Equation \ref{eq:VMSerr1D} that is 1D in nature is the parameter $\nu^{err}_{e,1D}$ because it is not immediately obvious how to choose $h_k$, the element mesh size, for multi-dimensional anisotropic elements.
This problem is equivalent to the problem found when generalizing the 1D stabilization parameter (that are used in stabilized finite element methods for advection-diffusion equations) to multi-dimensional problems \cite{hughes1986newiii} and can therefore be addressed with a similar approach.
%The generalization procedure follows three steps:

%\begin{enumerate}
%    \item Find the asymptotes of the expression for the advective and diffusive limits
%    \item Combine the asymptotes to get the final form
%    \item Employ the element metric tensor $g_{ij}$ in each limit 
%\end{enumerate}

%The asymptotes for $\nu^{err}_{k,1D}$ in Equation \ref{eq:VMSerr1D} are $\sqrt{\dfrac{h_k/2}{{\kappa|a_i|}} }$ for the advective limit and $\dfrac{ h_k/2 } {{\kappa\sqrt{3}}}=\sqrt{ \dfrac{(h_k/2)^2} {3\kappa^2}  }$ for the diffusive limit.
%They can then be combined through the following expression

%\begin{equation}
%    \nu^{err}_{k,1D} \approx \dfrac{1}{\sqrt{\dfrac{\kappa|a_i|}{h_k/2} + \dfrac{3\kappa^2}{(h_k/2)^2} } }
%\end{equation}

For advective-diffusive cases, there are two parts of interest: one corresponding to the direction of local advection and the other corresponding to diffusion.
This is reflected in the asymptotes for the exact expression for $\nu^{err}_{k,1D}$. %Equation \ref{eq:VMSerr1D}.
The magnitude of the local advection relative to the local mesh size, $\dfrac{|a_i|}{h_k/2}$, can be represented with $\sqrt{a_i g_{ij} a_j}$ (where $g_{ij} = \left(\dfrac{2}{h_k}\right)^2$ in 1D).
For the diffusive part, the element-level metric tensor intrinsically represents an average mesh size for an element so $g_{ij}g_{ij}$ provides a convenient choice.
This leads to the final generalized result for $\nu^{err}_k$

\begin{equation}
    \nu^{err}_{k} = \dfrac{1}{\sqrt{ \kappa\sqrt{a_i g_{ij} a_j} + 3\kappa^2\sqrt{g_{ij}g_{ij}}  }  }
\end{equation}

\noindent and the subsequent final form of the \textit{a posteriori} error estimator as follows

\begin{equation}
      |e^{\psi}_k|_{\HOne(\Omega^h_k)} = \seminorm{\psi'_k}_{H^1(\Omega^h_k)} \approx \nu^{err}_{k} \norm{\mathcal{R}(\bar{\psi}_k)}_{L^2(\Omega^h_k)}
      \label{eq:VMSerr}
\end{equation}

The effectivity of the VMS-based error estimator in 2D was tested by Zhang \cite{zhang19} for a 2D advection-diffusion problem. For this multi-dimensional case, it is reported that the error estimator does not exactly capture the true error. Instead, for a wide-range of $Pe_h$ the global effectivity was slightly greater than unity.   
For $Pe_h=100$, the effectivity was approximately 1.1, which indicates that the error estimator is slightly overestimating the true error.
Note that an effectivity above 1 implies a more conservative error estimator. 
For an advection-diffusion-reaction problem with a similar $Pe_h$, Ainsworth \textit{et al.} \cite{ainsworth2013fully} reports an effectivity of $\mathcal{O}(50)$ using the equilibrated residual method, which suggests that the VMS-based error estimator performs well.

 

