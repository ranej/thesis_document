%%%%%%%%%%%%%%%%%%%%%%%%%%%%%%%%%%%%%%%%%%%%%%%%%%%%%%%%%%%%%%%%%%% 
%                       rpithes-short.tex                         %
%         Template for a short thesis all in one file             %
%        (titlepage info below assumes masters degree}            %
%  Just run latex (or pdflatex) on this file to see how it looks  %
%      Be sure to run twice to get correct TOC and citations      %
%%%%%%%%%%%%%%%%%%%%%%%%%%%%%%%%%%%%%%%%%%%%%%%%%%%%%%%%%%%%%%%%%%% 
%
%  To produce the abstract title page followed by the abstract,
%  see the template file, "abstitle-mas.tex"
%
%%%%%%%%%%%%%%%%%%%%%%%%%%%%%%%%%%%%%%%%%%%%%%%%%%%%%%%%%%%%%%%%%%%

\documentclass{thesis}

% Uncomment the following if you want centered-lined captions:
%\captionsetup{format=plain,justification=centering}

%%%%%%%%%%%%%%%%%%%%  supply titlepage info  %%%%%%%%%%%%%%%%%%%%%
\thesistitle{\bf Differential Equations\\On two lines}
\author{Sir Isaac Newton}
%
% Select the appropriate degree
\degree{Master of Science}
%\degree{Master of Arts}
%\degree{Doctor of Philosophy}        
%\degree{Senior Thesis}
\department{Mathematics} % provide your area of study here; e.g.,
%  "Mechanical Engineering", "Nuclear Engineering", "Physics", etc.
%\signaturelines{1}     %max number of signature lines is 7, 1 is default
\thadviser{Galileo}
%\cothadviser{First co-adviser} %if needed
%\cocothadviser{Second co-adviser} % if needed
%  For a masters project use \projadviser instead of \thadviser, 
%  and \coprojadviser and \cocoprojadviser if needed. 
\submitdate{January 1685\\(For Graduation May 1685)}        
%\copyrightyear{1685}  % if date omitted, current year is used. 
%%%%%%%%%%%%%%%%%%%%%   end titlepage info  %%%%%%%%%%%%%%%%%%%%%%
      
\begin{document} 
\titlepage             % Print titlepage   
%\copyrightpage        % optional         
\tableofcontents       % required 
\listoftables          % required if there are tables
\listoffigures         % required if there are figures

\specialhead{ACKNOWLEDGMENT}
The acknowledgment text goes here. Unlike chapter headings, 
this heading is not numbered.

\specialhead{ABSTRACT}
Write your abstract here. Again, the heading does receive a number.

\chapter{INTRODUCTION}
The text of the first chapter goes here. To cite a reference for the
bibliography, use a command such as:\cite{thisbook}
\section{A Section Heading}
This is a sentence to take up space and look like text.
\subsection{A Subsection Heading}

\chapter{THE NEXT CHAPTER}
And so on, for more chapters.
Another citation for the bibliography:\cite{anotherbook}

% The following produces a numbered bibliography where the numbers
% correspond to the \cite commands in the text.
\specialhead{LITERATURE CITED}
\begin{singlespace}
\begin{thebibliography}{99}
\bibitem{thisbook} This is the first item in the Bibliography.
Let's make it very long so it takes more than one line.
Let's make it very long so it takes more than one line.
\bibitem{anotherbook} The second item in the Bibliography.
\end{thebibliography}
\end{singlespace}

%%%%%%%%%%%%%%%%%%%%%%%  For Appendices  %%%%%%%%%%%%%%%%%%%
\appendix    % This command is used only once!
\addtocontents{toc}{\parindent0pt\vskip12pt APPENDICES} %toc entry, no page #
\chapter{THIS IS AN APPENDIX}
Note the numbering of the chapter heading is changed.
This is a sentence to take up space and look like text.
\section{A Section Heading}
This is how equations are numbered in an appendix:
\begin{equation}
x^2 + y^2 = z^2
\end{equation} 

\chapter{THIS IS ANOTHER APPENDIX}
This is a sentence to take up space and look like text.

\end{document}
