\documentclass[letterpaper,11pt]{article}

\usepackage{calc}
\usepackage{array}
\usepackage{color}
\usepackage[margin=1in, left=1.25in, top=.5in, paperwidth=8.5in, paperheight=11in]{geometry}
\usepackage{enumitem}
\usepackage{times}
\usepackage{courier}
\usepackage[OT1, T1]{fontenc}

\usepackage{algorithm}
\usepackage[noend]{algpseudocode}
\usepackage{bm}
\usepackage{mathtools,commath}
\usepackage{placeins} %for floatbarrier
\usepackage{subcaption}
%\usepackage{multirow} %used for table formatting
\usepackage{tabularx}
\usepackage{array} % for table formatting to avoid hyphenation
\usepackage{siunitx} %for writing in scientific notation
\usepackage{csquotes} %for quotes in boundary marker section
\usepackage[hidelinks]{hyperref} %links to sections and figures



%\usepackage{fancyvrb}
%\usepackage{singlespace}

\definecolor{altncolor}{rgb}{0,0,.8}
%\definecolor{altncolor}{rgb}{0,0,0}

\usepackage[colorlinks=true, linkcolor=altncolor, anchorcolor=altncolor, citecolor=altncolor, filecolor=altncolor, menucolor=altncolor, urlcolor=altncolor]{hyperref}

\usepackage{parskip}


\lstset{basicstyle=\ttfamily, columns=fullflexible, keepspaces=true, xleftmargin=1em}

%\setlength{\parindent}{0pt}
%\setlength{\parskip}{.5\baselineskip}

\begin{document}

\title{Preparing a Thesis With \LaTeX}
\author{Client Information Services\\
Information Technology Infrastructure}
\date{September 2018}

\maketitle

\tableofcontents        

\newpage

\section{About the Thesis Class}

The Rensselaer {\LaTeX} thesis document class, available for download on the Web, can be used
to produce either a master's or a doctoral thesis with a format that meets the requirements of
the Office of Graduate Education.
The Thesis document class allows you to generate:
\begin{itemize}
   \item title page
   \item abstract title page
   \item copyright page (optional)
   \item lists of contents (table of contents, list of tables, and list of figures)
   \item acknowledgment, preface, etc.
   \item abstract
   \item chapters with numbered headings and subheadings
   \item bibliography
   \item appendices
\end{itemize}

Although the thesis document class produces an acceptable format, you should be familiar
with the regulations on thesis preparation produced by the Office of Graduate Education. The
OGE Thesis Writing Manual, which includes information on electronic submission, is available
online at
\url{https://info.rpi.edu/sites/default/files/OGE Preparation Manual.pdf}
% \url{http://www.rpi.edu/dept/grad/docs/ThesisGuide/manual.pdf}

% These directions need updating.
\href{https://dotcio.rpi.edu/services/printing-publishing/thesis-preparation}{The Rensselaer {\LaTeX} Thesis web page}
provides information on writing a thesis with {\LaTeX},
including documentation and files for download.
%You can reach this page easily from the Help
%Desk web page: in the Quick Links drop-down menu, select ``Prepare a thesis''.
If you are not
familiar with {\LaTeX}, first read the ARC tutorial,
\href{http://www.rpi.edu/dept/arc/docs/latex/latex-intro.pdf}{Text Formatting with {\LaTeX}},
which will get you started. You can print it from the Thesis web page, or pick it up
free of charge at the VCC Help Desk.

Another good source of information is the Rensselaer {\LaTeX} information web page,
\url{http://www.rpi.edu/dept/arc/training/latex/}. In addition to course material, this page
has links to various useful documents and a number of {\LaTeX} examples.

The complete reference for {\LaTeX} is the \textit{{\LaTeX} User's Guide} by Leslie Lamport. Since the
thesis class is based on the standard {\LaTeX} class \textbf{report}, the information in this book and
in the ARC tutorial, \textit{Text Formatting with {\LaTeX}}, applies to using the thesis class as well as
standard classes. Any differences are described in this document.

The thesis class and the template files described in this document can be used on any system
with {\LaTeX}, which includes Windows machines, Macs, or Linux/Unix systems. It is assumed
that you know how to run {\LaTeX} on your system. Note, however, that using RCS Unix is
not recommended because it is no longer actively maintained, and its {\LaTeX} programs and
packages are not recent enough.

\section{Downloading the Thesis Class}

To use the thesis class on your personal computer, you first need to download the file \verb|thesis.cls|
from the \href{https://dotcio.rpi.edu/services/printing-publishing/thesis-preparation}{{\LaTeX} Thesis web page}.
You can either put it in the folder with your thesis material
(to use it only with documents in that folder), or place it in the standard TEX input path
for your system, along with all the other files that end in .cls or .sty. If you do the latter, it's best to put thesis.cls in a folder you create in a ``local tree'' to preserve it in the
case of future TeX upgrades. It is easiest to create the folder first and then download the file
thesis.cls. For example, If you are using the TeXLive system on a Windows machine and
% HERE: Update years, and check installation locations.
have installed under \verb|C:\TeXLive2005|, the folder name will be:
\verb|C:\TeXLive2005\texmf-local\tex\latex\thesis\|.

After downloading the file, check that Windows has not named it
\verb|thesis.cls.txt| instead of \verb|thesis.cls|! Then, if you've chosen to put it
in \TeX's input path, be sure to rebuild the ``\verb|ls-R|'' filename
database: for a {\TeX}Live installation on Windows, go to Start \verb|->|
Programs \verb|->| TeXLive and follow the appropriate links from
there.

\section{Using the Template Files}

For a quick start, you can use the template (or prototype) files. You
can download these template files from links on the
\href{http://www.rpi.edu/dept/arc/docs/latex-thesis/}{{\LaTeX} Thesis web page}
or directly from
\url{http://www.rpi.edu/dept/arc/docs/latex-thesis/}. Put them in your own
folder or directory that will hold the material for your thesis and
from which you will be running {\LaTeX}.  Note: To create correct
tables of contents and to resolve forward references, remember that
you need to run {\LaTeX} or pdf{\LaTeX} twice. This is necessary
because the information taken from the auxiliary files, which store
this information, is always from the previous run.

\subsection{Short Thesis}

If your thesis is only about 20--30 pages, you will probably want to keep everything in one file.
In this case, you can download the template file for a short thesis:
\begin{verbatim}
rpithes-short.tex
\end{verbatim}
Replace the text with your own, and run \verb|latex| or \verb|pdflatex| to produce your thesis. A listing
of the above file is in is in \nameref{sec:appA} of this document. You'll also need the template for
producing the abstract title page and abstract, a separate file required by OGE for all theses:
\begin{verbatim}
abstitle-mas.tex
\end{verbatim}
\nameref{sec:appB} lists this template file and the corresponding file for a Ph.D. dissertation.

\subsection{Long Thesis}

For a longer thesis or dissertation, it will be easier to use separate files for different sections.
The set of template files below are designed to do this: each file contains the commands to
produce a particular section. The root file, rpithes.tex, is the file that you supply as input
to the {\LaTeX} (or pdf{\LaTeX}) program, and it in turn instructs {\LaTeX} to process the other files.
\begin{verbatim}
rpithes.tex (root file)     rpiack.tex     rpichap2.tex
rpititle-mas.tex (masters)  rpiabs.tex     rpibib.tex
rpititle-phd.tex (Ph.D.)    rpichap1.tex   rpiapp.tex
\end{verbatim}
In addition to your thesis or dissertation, you are required to submit a separate file containing
just the abstract title page and the abstract. You can prepare this file using one of templates:
\begin{verbatim}
abstitle-mas.tex    abstitle-phd.tex
\end{verbatim}


These two template files are listed in \nameref{sec:appB}. The listings of the set of template files for
a longer thesis are in \nameref{sec:appC}, and the output they produce is included as \nameref{sec:appD}.
The root file, rpithes.tex, is also reproduced below.
After downloading the template files, you will probably want to choose your own names for
these files. If you change the file names, be sure to edit the ``root file'' rpithes.tex and change
the \verb|\include| statements accordingly. You can then modify the text of the template files and
run latex or pdflatex on the root file.

\lstinputlisting{rpithes.tex}


By using a root file with \verb|\include| commands, you can produce the entire thesis, or you can
use the \verb|\includeonly| command to produce just certain parts. In nthe prototype file above,
this command instructs {\LaTeX} to process only the file rpichap1.tex. To process more than one
file, include several file names (separated by commas) as the argument to the \verb|\includeonly|
command. For example, in the prototype file, the following command would instruct {\LaTeX} to
process only the files rpititle-mas.tex and rpiabs.tex.
\begin{verbatim}
\includeonly{rpititle-mas,rpiabs}
\end{verbatim}
To process the entire thesis, comment out the \verb|\includeonly| command by preceding it with a
percent sign (\%).


\section{Thesis Document Class Options}

Document class options, which are specified in square brackets on the
\verb|\documentclass| command, provide various modifications to the
formatting of the text. There are several options you may find useful
with the thesis class.

\subsection{Type Size}

By default, the thesis is in 12-point type. Two smaller type sizes, 10 and 11 points, are available
as options on the documentclass command. For example, to use 11 points, edit the root file
(i.e., rpithes.tex) and specify
\begin{verbatim}
\documentclass[11pt]{thesis}
\end{verbatim}


\subsection{The Chapter Heading Format}


Another option, this one unique to the thesis class, is chap. The chap option writes the word
``CHAPTER'' on a separate line above the chapter title. If you have included \verb|[chap]| in the
documentclass command, the line
\begin{verbatim}
\chapter{INTRODUCTION AND HISTORICAL REVIEW}
\end{verbatim}
would produce:
\begin{center}
  {\large\bf CHAPTER 1
    
  INTRODUCTION AND HISTORICAL REVIEW
  }
\end{center}
Without the chap option, the chapter title would look like:
\begin{center}
  {\large\bf 1. INTRODUCTION AND HISTORICAL REVIEW}
\end{center}
Note that if you use two or more options, you must separate them with commas. Therefore,
to use both the chap option and the 11-point option, use the command:
\begin{verbatim}
\documentclass[chap,11pt]{thesis}
\end{verbatim}


\subsection{Twosided Formatting}

Although the Office of Graduate Education requires a onesided copy of your thesis, your
department or other recipients may be happy with a double-sided copy. If you include twoside
in square brackets in the \verb|\documentclass| command, your thesis will be formatted for twosided
printing. This means that the 1.5 inch margin, which is always the left margin on onesided
pages, will cycle so that it is always on the binding edge, and that page numbers, normally in
the upper right corner, will cycle so that they are always on the outside edge. (This option
does not force the printer to print double-sided. To also get twosided printing, you must use a
duplex printer, such as VCLW.)

Using the twoside option will also ensure that the table of contents does not print on the back
of the title page and that the first chapter always starts on a right-hand page. Subsequent
chapters, however, will not necessarily begin on a new sheet of paper. To force each chapter
to start on a right-hand page, also include the openright option:
\begin{verbatim}
\documentclass[twoside,openright]{thesis}
\end{verbatim}


\section{Other Features and Considerations of the Thesis Class}

The thesis document class contains some features that are not part of the standard {\LaTeX}
classes. Most of these are built into the thesis class; a few are provided by \emph{packages}, sets of
{\LaTeX} or {\TeX} commands written by users and made available to the {\LaTeX} community. A
package often defines totally new commands that add extra features.

\subsection{Producing Unnumbered Section Headings and Appendices}

The command \verb|\specialhead| produces a section heading similar to those produced by the
\verb|\chapter| command but without a number. Use it for Abstract, Acknowledgment, Bibliography, etc.

Note that appendices are produced with the \verb|\chapter| command, but you must have previously
included the {\LaTeX} command \verb|\appendix|. (This is documented in the \textit{{\LaTeX} User's Guide} and
illustrated in the RPI template file \verb|rpiapp.tex|.) Note the \verb|\appendix| command should appear
only once, before the first appendix. (Do NOT include it before each appendix.)

\subsection{Footnote Numbering}

Unlike other {\LaTeX} document classes, which reset the footnote counter to 1 at the start of each
new chapter, the thesis class numbers footnotes sequentially throughout the thesis. To start
over with number~1 at any time, use the command \verb|\resetfootnote|.

\subsection{Figure and Table Captions}

Prepare figures and tables using the figure and table environments as
described in the \textit{{\LaTeX} User's Guide}, and use the \verb|\caption|
command to specify the caption.\footnote{Note: if you are planning to
  cross-reference the caption, be sure to put the \label command after
  the caption.}  In the thesis class, table and figure captions are in
boldface type by default.  Short captions are centered on the line;
captions longer than one line are left-aligned. If you would like to
indent subsequent lines of long captions, you can use the caption
package with the hang option to do this. After the
\verb|\documentclass| command, include the line:
\begin{verbatim}
\usepackage[hang]{caption}
\end{verbatim}
If you use the caption package, your captions will not be bold. However, the package provides
the captionfont command, which allows you to control the font of the captions. Therefore,
to get bold with the caption package, follow the above command with the line:
\begin{verbatim}
\renewcommand{\captionfont}{\bfseries}
\end{verbatim}
Note that the above two lines should be part of your preamble–that is, after the \verb|\documentclass|
command and before the \verb|\begin{document}| command. See the root template file rpithes.tex
for an example.

\subsection{Line Spacing}

The spacing of your thesis will be line-and-a-half, which is acceptable to the Office of Graduate
Education. This spacing was achieved by using a stretch factor of 1.4, which is just right for
typesizes of 12 points (the default) and 11 points. If you choose 10 points, the smallest type size
allowed, you should increase the spacing slightly by including in your preamble the command
\verb|\setstretch{1.5}|.

The thesis document class defines a new environment called singlespace. To single space a
section of text inside the otherwise line-and-a-half-spaced thesis, do the following:
% Here: How to get the text in italic inside verbatim?
\begin{verbatim}
\begin{singlespace}
put the single-spaced text here
\end{singlespace}
\end{verbatim}

\subsection{Heading Size}

If you wish, you can change the type size of your section headings. By default, the chapter
and section headings are a little larger than the text, and the subsection and subsubsection
headings are the same size as the text. (All headings are boldface.) Below are the heading
size commands used by default. You can change any of them by putting a similar command
in your preamble with a different size specified.
\begin{verbatim}
\renewcommand\chaptersize{\large}
\renewcommand\sectionsize{\large}
\renewcommand\subsectionsize{\normalsize}
\renewcommand\subsubsectionsize{\normalsize}
\end{verbatim}

\subsection{The Bibliography}

\subsubsection{Using {\LaTeX}'s Built-in Method}


To prepare a bibliography in {\LaTeX}, you use the command \verb|\cite{key}|
within your text to cite various works. ``key'' is a keyword of your
choosing that identifies the work. For example your document might
include, at the appropriate places: \verb|\cite{lamport}| \verb|\cite{kopka}|
\verb|\cite{goossens}|. These commands place numbers (enclosed in square
brackets) in the text that match the numbers which will be
automatically generated in the bibliography. (Remember to run {\LaTeX}
twice to get correct numbers in the text!) Then, at the end of the
document, you put your bibliographic entries in a special environment
called \textbf{thebibliography}.
\href{http://www.rpi.edu/dept/arc/docs/latex/latex-intro.pdf}{Text Formatting with {\LaTeX}}
has more
information on preparing a bibliography.

This method is illustrated
in the sample thesis appended to this document. Note that the entries
in the template file \texttt{rpibib.tex} are inside the \texttt{singlespace}
environment. This produces an attractive bibliography and is
recommended, though certainly not required.

The alignment of the
bibliography section is ragged right by default, because in many cases
it looks better. (When fully justified, a bibliography can have some
very wide spaces between words.) However, if you prefer that it be
fully justified, just put the following command in the preamble:
\verb|\renewcommand{\bibalign}{}|

\subsubsection{Using Bib{\TeX} with the Thesis Class}

Bib{\TeX}, a separate program included with {\TeX} distributions, generates a list of references
from information contained in a bibliographic database—a file you create whose name ends
with the extension .bib. the There are several books, inluding Leslie Lamport's {\LaTeX} manual,
that describe in detail how to use Bib{\TeX} and how to prepare the
\verb|.bib| file.

If you use one of the basic bibliography styles such as plain, unsrt or alpha, using Bib{\TeX}
with the thesis class is straightforward. In your \verb|rpibib.tex| file, use \verb|\specialhead| to make
an unnumbered heading. Then add the \verb|bibliographystyle| command and the \verb|biblography|
command. For example, if the section heading is ``REFERENCES,'' if you are using the \verb|unsrt|
bibliography style, and if your database entries are in the file \verb|myrefs.bib|, your \verb|rpibib.tex|
file would look like:
\begin{verbatim}
\specialhead{REFERENCES}
\bibliographystyle{unsrt} % specify bibliography style
\begin{singlespace}
\bibliography{myrefs} % Prints the bibliography here, using "myrefs.bib"
\end{singlespace}
\end{verbatim}
That's all. Just remember that to create the bibliography, you must run {\LaTeX}, then Bib{\TeX},
then run {\LaTeX} twice more. Windows editors, such as WinShell and WinEdt, have a button
for Bib{\TeX} on the toolbar.

If you are further customizing your bibliography by using a package
such as natbib\footnote{A good overview of how to use natbib is at
  \url{http://www.ctan.org/tex-archive/macros/latex/contrib/natbib/natnotes.pdf}}
or harvard, do not use \verb|\specialhead|. The package will make its own new
page and heading, and you don't want two! But you will need to add the
command \verb|\addcontentsline| to get the entry into the table of
contents. And, if you are using the hyperref package to put live links
in your PDF file, you'll also need the \verb|\phantomsection| command
to put the anchor in the right place. Assuming you want the title to
be ``REFERENCES'' (rather than the default name ``BIBLIOGRAPHY''),
your preamble would include commands such as:
\begin{verbatim}
\usepackage{harvard}
\renewcommand\bibname{REFERENCES} % specify name of your heading
\end{verbatim}
and your rpibib.tex file might look like:
\begin{verbatim}
\clearpage
\phantomsection % To make hyperref link in TOC work correctly
\addcontentsline{toc}{chapter}{\bibname} % puts entry in TOC.
\bibliographystyle{agsm} % specify bibliography style
\begin{singlespace}
\bibliography{myrefs} % Prints the bibliography here, using "myrefs.bib"
\end{singlespace}
\end{verbatim}


\subsection{Making an Index}

An index is not required for your thesis, but you can include one if
you would like to.
\href{http://www.rpi.edu/dept/arc/docs/latex/latex-intro.pdf}{Text Formatting with {\LaTeX}}
includes a section on generating an index, which describes what you
need in the preamble and how to index the entries. For more complete information, the
documentation that comes with the makeindex program, makeindex.dvi, should be available
on your system.

To print an index at the end of your thesis, there are several commands you will want to
use in addition to the usual \verb|\printindex| command. You'll want the Index in the Table of
Contents, and you'll want single spacing. You do not want to use \verb|\specialhead| because
makeindex automatically creates the heading, and you don't want two. And, if you are using
the hyperref package to put live links in your PDF file, you'll also need the \verb|\phantomsection|
command to put the anchor in the right place. You can put these commands in a separate file
which you \verb|\include| in your root file. A file named, for example
\verb|rpiind.tex|, might look like:
\begin{verbatim}
\clearpage
\phantomsection
% To make hyperref link in TOC work correctly
\addcontentsline{toc}{chapter}{\indexname} % puts entry in TOC
\begin{singlespace}
\printindex
\end{singlespace}
\end{verbatim}
Remember that you must run {\LaTeX} (or pdf{\LaTeX}), then \verb|makeindex|, then {\LaTeX}(or pdf{\LaTeX})
again. WinEdt has a menu item to run makeindex, but other Windows editors may not. If you
don't have a menu item, you'll need to open a command window and \verb|cd| to the appropriate
directory/folder) to run \verb|makeindex|.

\section{Your Final Output: Creating the PDF Files}

Rensselaer requires that electronically-submitted theses or
dissertations be in Adobe Portable Document Format (PDF), the current
standard for electronic information exchange. PDF files look exactly
like the original documents and are viewable and printable on any
platform. Remember that you need to make two PDF files: one containing
the complete thesis or dissertation and the other containing just the
Abstract Title Page and abstract.

After you have written your thesis in {\LaTeX}, it is straightforward to convert your .tex files to
PDF. There are two widely-used methods:
\begin{enumerate}
\item The traditional way is to run {\LaTeX} followed by dvips to
  create a PostScript file and then convert that to PDF. On Windows,
  you can do the conversion by opening the .ps file with GSView and
  using menu items to convert to PDF; on unix/Linux systems, you run
  the \verb|ps2pdf| program (part of \verb|ghostscript|).

\item A simpler method is to use the relatively recent program
  pdf{\LaTeX}, which processes your {\LaTeX} file and produces a PDF
  file directly. On Windows systems, your editor/shell (e.g.,
  WinShell, WinEdt) has a pdf{\LaTeX} button on the menu bar; on unix
  systems you type \verb|pdflatex <filename>| on the command
  line.\footnote{Acrobat Reader cannot automatically update the view
    if you reprocess your document, unlike xdvi and GSview. You have
    to close the display with \texttt{Ctrl-W} and reload the file with \texttt{Alt-}{$\leftarrow$}
    (left arrow). Or, you can configure your editor to view PDF files
    with GSView instead of Acrobat.}
\end{enumerate}
This is simple enough. Unfortunately, a complication arises when you
consider included graphics.
(See section~\ref{sec:graphics}
for how to include graphics
in a {\LaTeX} file.) When you use the traditional conversion method
({\LaTeX} plus \verb|dvips|), your graphics files must be in \textbf{eps}
(Encapsulated PostScript) format. But when you use pdf{\LaTeX}, it
accepts the formats \textbf{pdf}, \textbf{jpg}, and \textbf{png}, but not \textbf{eps}. If you want to use
pdf{\LaTeX} and your graphics files are in \textbf{eps} format, a solution is
to convert them to \textbf{pdf} using the \verb|epstopdf| utility, which is most
likely on your system.

For detailed instructions on creating the PDF files, including how to manage graphics files and
how to make hyperlinks, see Creating a PDF File from a {\LaTeX} Thesis, at
\url{http://www.rpi.edu/dept/arc/docs/latex-thesis/latextopdf.pdf}.


\section{Including Graphics}
\label{sec:graphics}

You can use a variety of applications to create your graphics. Maple, Matlab, CorelDRAW,
Xfig, Gnuplot, and even Windows applications such as Word or Excel are common choices. If
you are using {\LaTeX} plus dvips to produce your final output, you should save the graphic as
encapsulated PostScript (\textbf{EPS}), not plain PostScript. If you are using pdf{\LaTeX}, you can save
it as \textbf{pdf}, \textbf{jpg}, or \textbf{png}.
To import graphics, you first need to load the graphicx package in
your preamble:
\begin{verbatim}
\usepackage{graphicx}   % Note the "x" in "graphicx"
\end{verbatim}
And then, at the spot you want to insert the graphic (for example, myfigure.eps or myfigure.pdf)
use the \verb|\includegraphics| command:
\begin{verbatim}
\includegraphics{myfigure}  % note filename extension is omitted
\end{verbatim}

If you want to be able to use either pdflatex or latex on the same file, you'll need to have your
graphics files in both eps format and one of the others, such as pdf. Then omit the filename
extension on the \verb|\includegraphics| command: latex will look for an eps file, and \verb|pdflatex|
will look for a \textbf{pdf}, \textbf{jpg}, or \textbf{png} file. The \verb|\includegraphics| command also provides optional
arguments for scaling or rotating the figure. Assuming you have a file named \verb|myfigure.eps|
or \verb|myfigure.pdf| (or both), the command, which usually goes inside the figure environment,
will look something like:
\begin{verbatim}
\includegraphics[width=4in]{myfigure}
\end{verbatim}
There is more information on the \verb|\includegraphics| command in
\href{ttp://www.rpi.edu/dept/arc/docs/latex/latex-intro.pdf}{Text Formatting with {\LaTeX}}.
Official documentation for the graphics package is in the file \verb|grfguide.pdf|; look for it on your
system. The information on including graphics is in Section~4.4.

After you have put the appropriate commands for including graphics into your {\LaTeX} file, you
can run pdf{\LaTeX} and view the result with Acrobat or GSView. If you create a \textbf{.dvi} file by
running {\LaTeX} and then view it with your previewer, most of the time the previewer will be able
to display the included PostScript graphics (by calling the \verb|ghostscript| program). However,
there may be some cases, for example if the graphic is in landscape orientation, where it is
not displayed properly. In this case, you can use dvips to put the output in a PostScript file
which will display the result correctly.

There is a wealth of information in \textit{Using Imported Graphics in {\LaTeX}2e}, a PDF document by
Keith Reckdahl of Stanford University. It includes all you would ever want to know with many
examples. You can find it at:
\url{http://www.ctan.org/tex-archive/info/epslatex.pdf}.


\section{Printing Landscape Figures and Tables}

You can print figures or tables, along with their captions, in landscape orientation (sideways)
through the use of the rotating package. To use this package, put the following command in
your preamble:
\begin{verbatim}
\usepackage{rotating}
\end{verbatim}
The rotating package defines two new environments, \textbf{sidewaysfigure} and \textbf{sidewaystable},
which can be used in place of the standard {\LaTeX} environments \textbf{figure} and \textbf{table}.
Probably the easiest way to see how to insert figures (either portrait or landscape) or landscape
tables is to look at examples. The file \verb|exrotating.tex|, one of the example files on the
{\LaTeX} information web page, contains examples of using both the \textbf{graphicx} and the \textbf{rotating}
packages. You can view the \verb|.tex| file to see the {\LaTeX} commands and the file \verb|exrotating.ps|
or \verb|exrotating.pdf| to see the results.

You can run this file yourself by copying exrotating.tex to your own space, along with the
\verb|.eps| and/or \verb|.pdf| files it uses. If you process it with pdf{\LaTeX}, the resulting \textbf{PDF} file should
look fine. If you process it with {\LaTeX}, note that you should create a \verb|.ps| file to view the result,
as most dvi previewers cannot display landscape graphics or tables properly.

\appendix
%\setcounter{secnumdepth}{0}

% HERE: Using * also takes them out of the TOC.

\newpage

\section*{Appendix A}
\label{sec:appA}
\addcontentsline{toc}{section}{\nameref{sec:appA}: Template File for a Short Thesis}

\subsection*{Template File for a Short Thesis}

Filename: \verb|rpithes-short.tex|

\lstinputlisting{rpithes-short.tex}

\newpage

\section*{Appendix B}
\label{sec:appB}
\addcontentsline{toc}{section}{\nameref{sec:appB}: Template Files for Abstract Title Page and Abstract}

\subsection*{Template Files for Abstract Title Page and Abstract}

Filename: \verb|abstitle-mas.tex|

\lstinputlisting{abstitle-mas.tex}

Filename: \verb|abstitle-phd.tex|

\lstinputlisting{abstitle-phd.tex}

\newpage

\section*{Appendix C}
\label{sec:appC}
\addcontentsline{toc}{section}{\nameref{sec:appC}: Template Files for Longer Thesis}

\subsection*{Template Files for Longer Thesis}

Filename: \verb|rpithes.tex|

\lstinputlisting{rpithes.tex}

Filename: \verb|rpititle-mas.tex|

\lstinputlisting{rpititle-mas.tex}

Filename: \verb|rpititle-phd.tex|

\lstinputlisting{rpititle-phd.tex}

Filename: \verb|rpiack.tex|

\lstinputlisting{rpiack.tex}

Filename: \verb|rpiabs.tex|

\lstinputlisting{rpiabs.tex}

Filename: \verb|rpichap1.tex|

\lstinputlisting{rpichap1.tex}

Filename: \verb|rpichap2.tex|

\lstinputlisting{rpichap2.tex}

Filename: \verb|rpibib.tex|

\lstinputlisting{rpibib.tex}

Filename: \verb|rpiapp.tex|

\lstinputlisting{rpiapp.tex}

\newpage

\vspace*{3in}

\section*{Appendix D}
\label{sec:appD}
\addcontentsline{toc}{section}{\nameref{sec:appD}: Output from the Template Files for Longer Thesis}

\subsection*{Output from the Template Files for Longer Thesis}

\includepdf[pages=-]{rpithes.pdf}


\end{document}
